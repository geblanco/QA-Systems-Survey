\documentclass{beamer}
\usepackage{pgfpages}
\usepackage{hyperref}
\usepackage[
backend=biber,
style=alphabetic,
citestyle=authoryear
]{biblatex}

\addbibresource{library.bib}

% \setbeameroption{show only notes}
\usetheme{material}

% \usePrimary{00897B}{00695C}{FFFFFF}
\usePrimaryTeal
% \useAccent{009688}{000000}
\useAccent{004D40}{000000}

\title{Core techniques of \\ \textbf{QA Systems over KBs \\ a Survey}}
\author{Guillermo Echegoyen Blanco}
\date{}

\begin{document}

\maketitle

% ToDo := TOC

\begin{frame}{Summary}
  \begin{card}
    Here goes the Summary
  \end{card}
\end{frame}

\begin{comment}
\begin{frame}{Tasks}
  \centering
  \begin{columns}
    \setbeamercolor{coloredboxstuff}{fg=white,bg=white!10!box1}
    \begin{beamercolorbox}[wd=0.33\textwidth,sep=1em]{coloredboxstuff}
      \centering
      Question Analysis
    \end{beamercolorbox}
    \setbeamercolor{coloredboxstuff}{fg=white,bg=white!10!box2}
    \begin{beamercolorbox}[wd=0.33\textwidth,sep=1em]{coloredboxstuff}
      \centering
      Phrase Mapping
    \end{beamercolorbox}
    \setbeamercolor{coloredboxstuff}{fg=white,bg=white!10!box3}
    \begin{beamercolorbox}[wd=0.33\textwidth,sep=1em]{coloredboxstuff}
      \centering
      Disambiguation
    \end{beamercolorbox}
  \end{columns}
  \begin{columns}
    \setbeamercolor{coloredboxstuff}{fg=white,bg=white!10!box4}
    \begin{beamercolorbox}[wd=0.5\textwidth,sep=1em]{coloredboxstuff}
      \centering
      Query Construction
    \end{beamercolorbox}
    \setbeamercolor{coloredboxstuff}{fg=white,bg=white!10!box5}
    \begin{beamercolorbox}[wd=0.5\textwidth,sep=1em]{coloredboxstuff}
      \centering
      Distributed Knowledge
    \end{beamercolorbox}
  \end{columns}
\end{frame}
\end{comment}

\begin{frame}{Intro}
  \begin{card}
    \begin{itemize}
      \item A Question Answering System should be able to: \\
      \textit{Understand a Natural Language Question so as to be able to answer based on some pre-known data.}
    \end{itemize}
  \end{card}
  \begin{card}
    \begin{itemize}
      \item Typically involves accepting a question and generating a SparQL query capable of extracting the information which answers the user question.
    \end{itemize}
  \end{card}
\end{frame}

\begin{frame}{Tasks}
  \begin{card}
    \begin{itemize}
      \item Question Analysis
      \item Phrase Mapping
      \item Disambiguation
      \item Query Construction
      \item Distributed Knowledge
    \end{itemize}
  \end{card}
\end{frame}

\begin{frame}{Question Analysis \#1}
  \begin{card}
    Analyze syntactic features to extract meaningful information:
    \begin{itemize}
      \item Type of question (is it a Which, What\dots question).
      \item Multilinguality (is it in English, French\dots).
      \item Correspondance to KB entities/classes.
      \item Tokens in the sentence and it's relations.
      \item Useless words in the sentence.
    \end{itemize}
  \end{card}
\end{frame}

\begin{frame}{Question Analysis \#2}
  \begin{card}
    Techniques based on:
    \begin{itemize}
      \item Recognizing Named Entities
      \item Segmenting with $POS^{*}$ Tags
      \item Identifying dependencies using parsers
    \end{itemize}
  \end{card}
  \vspace{8em}
  POS Tag: Part-Of-Speech Tag
\end{frame}

\note[itemize]{
  % ToDo := Correct this notes
  \item Recognizing Named Entities consists in finding the entities corresponding to parts of the phrase (eg: Europe dbr:European\_Union):Which token correspond to which resource in the KB
  \item Segmenting is like tokenization of different parts of the string, where the tag is usually universal
  \item Dependencies refer to parts of the phrase which depend upon others, direct cumpliment, adjective, subjective noun\dots
}

\begin{frame}{Question Analysis \#3 - Recognizing named entities}
  \begin{card}
    Identify Named Entities and map to resource in KB
    \begin{itemize}
      \item \textbf{\textit{NER} Tools}: Tools from NLP, \textbf{\textit{Standford NER Tool}}. Domain specific, low precision 51\% (\cite{he2014a})
      \item \textbf{\textit{N-Gram}}: Map n-grams to KB entities. Adv: Each NE can be recognized in the KB. Disadv: Dissambiguation explodes (too much candidates). (SINA: \cite{shekarpour2015a}, CASIA: \cite{he2014a})
      \item \textbf{\textit{Entity Linking} Tools}: \textbf{DBpedia Spotlight} (\cite{daiber2013a}) and \textbf{AIDA} (\cite{yosef2011a}). Recognize NE and find the underlying KB resource, dissambiguating on the way. Adv: All-in-one. Disadv: Limited service, KB dependant.
    \end{itemize}
  \end{card}
\end{frame}

\note{
  Identify tokens in the sentence that refer to a resource in the KB, discarding useless words.
  \begin{itemize}
    \item When grouping n-grams, if an entity is found, the n-gram is considered, else more n-grams are tried.
  \end{itemize}
}

\begin{frame}{Question Analysis \#4 - Segmenting using POS Tagging}
  \begin{card}
    Identify which phrase correspond to instances, properties, classes\dots and which is irrelevant.
  \end{card}
\end{frame}

\begin{frame}{Phrase Mapping}
\end{frame}
\begin{frame}{Disambiguation}
\end{frame}
\begin{frame}{Query Construction}
\end{frame}
\begin{frame}{Distributed Knowledge}
\end{frame}

\begin{frame}{Bibliography}
  \printbibliography
\end{frame}
% ToDo := Bibliography

\end{document}
